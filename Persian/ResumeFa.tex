\documentclass[10pt, a4paper]{article}

% Packages:
\usepackage[utf8]{inputenc}
\usepackage{fontspec} % For custom fonts


\usepackage[
    ignoreheadfoot, 
    top=2 cm,
    bottom=2 cm,
    left=2 cm,
    right=2 cm,
    footskip=1.0 cm
]{geometry} % for adjusting page geometry
\usepackage{titlesec} % for customizing section titles
\usepackage{tabularx} % for making tables with fixed width columns
\usepackage{array} % tabularx requires this
\usepackage[dvipsnames]{xcolor} % for coloring text
\usepackage{enumitem} % for customizing lists
\usepackage{amsmath} % for math
\usepackage{hyperref} % for links, metadata and bookmarks
\usepackage{lastpage} % for getting the total number of pages
\usepackage{changepage} % for one column entries (adjustwidth environment)
\usepackage{needspace} % for avoiding page break right after the section title
\usepackage{iftex} % check if engine is pdflatex, xetex or luatex
\usepackage{xepersian} % For Persian language support

\settextfont{XB Niloofar}
\setdigitfont[Scale = 1.0,
BoldFont = * Bd,
ItalicFont = * It,
BoldItalicFont = * BdIt,
Extension = .ttf
]{Yas}

% Set RTL layout
\setRTL

% Some settings:
\raggedleft
\AtBeginEnvironment{adjustwidth}{\partopsep0pt} % remove space before adjustwidth environment
\pagestyle{empty} % no header or footer
\setcounter{secnumdepth}{0} % no section numbering
\setlength{\parindent}{0pt} % no indentation
\setlength{\topskip}{0pt} % no top skip
\setlength{\columnsep}{0.15cm} % set column separation
\pagenumbering{gobble} % no page numbering

\titleformat{\section}{\needspace{4\baselineskip}\bfseries\large}{}{0pt}{}[\vspace{1pt}\titlerule]

\titlespacing{\section}{
    -1pt
}{
    0.3 cm
}{
    0.2 cm
}

\renewcommand\labelitemi{$\vcenter{\hbox{\small$\bullet$}}$} % custom bullet points

% Define missing environments
\newenvironment{header}{
    \centering
}{\par}

\newenvironment{twocolentry}[1]{
    \begin{tabularx}{\textwidth}{Xl} % Change to 'Xl' for RTL
}{\end{tabularx}}

\newenvironment{onecolentry}{
    \begin{adjustwidth}{0pt}{0pt}
}{\end{adjustwidth}}

\newenvironment{highlights}{
    \begin{itemize}[rightmargin=*] % Align itemize to right
}{\end{itemize}}

% Begin Document
\begin{document}

    % Header
    \begin{header}
        \fontsize{25 pt}{25 pt}\selectfont عرفان حبیبی احسنی

        \vspace{5 pt}

        \normalsize
        تهران، ایران%
        \kern 5.0 pt%
        \textbar%
        \kern 5.0 pt%
        \href{mailto:erfan.habibi.ehsani@gmail.com}{erfan.habibi.ehsani@gmail.com}%
        \kern 5.0 pt%
        \textbar%
        \kern 5.0 pt%
        \href{https://www.linkedin.com/in/erfanhabibi/}{linkedin.com/in/erfanhabibi}%
        \kern 5.0 pt%
        \textbar%
        \kern 5.0 pt%
        \href{https://github.com/Erfanhabibi}{github.com/Erfanhabibi}%
    \end{header}

\vspace{10 pt}

% Education Section
\section{تحصیلات}

\begin{tabularx}{\textwidth}{Xl} % Change to 'Xl' for RTL
    \textbf{دانشجو کارشناسی در ریاضیات و کاربردها} & دانشگاه امیرکبیر، تهران، ایران \\
\end{tabularx}

\vspace{0.1 cm}

\textbf{تاریخ فارغ‌التحصیلی :} 1404\\

\textbf{دوره‌های مرتبط:} ساختارهای داده، الگوریتم‌ها، یادگیری ماشین، برنامه‌نویسی پیشرفته

% Experience Section
\section{تجربیات}

\begin{tabularx}{\textwidth}{Xl} % Change to 'Xl' for RTL
    \textbf{پروژه‌های شخصی و آکادمیک} & 2021 -- حال \\ 
\end{tabularx}

\vspace{0.1 cm}

\textbf{نکات کلیدی:}
\begin{itemize}
    \item توسعه یک برنامه پایتون با استفاده از OpenCV برای پردازش تصویر در زمان واقعی، با امکاناتی از جمله تشخیص شی و فیلتر کردن تصویر.
    \item پیاده‌سازی یک پروژه داده‌کاوی با استفاده از پایتون، استخراج اطلاعات ارزشمند از مجموعه‌های داده بزرگ با استفاده از کتابخانه‌هایی مانند pandas و NumPy برای تصمیم‌گیری.
    \item ایجاد برنامه‌های وب تعاملی با استفاده از جاوا و جاوااسکریپت، افزایش تجربه کاربری از طریق محتوای پویا و طراحی واکنش‌گرا.
\end{itemize}

% Projects Section
\section{پروژه‌ها}

\begin{tabularx}{\textwidth}{Xl} % Change to 'Xl' for RTL
    \textbf{سیستم ارتباط سرور-کلاینت} & \href{https://github.com/Erfanhabibi/python-socket.git}{github.com/Erfanhabibi/python-socket} \\ 
\end{tabularx}

\vspace{0.1 cm}

\textbf{نکات کلیدی:}
\begin{itemize}
    \item پیاده‌سازی یک سیستم ارتباط سرور-کلاینت با استفاده از سوکت‌های پایتون برای انتقال داده‌های تصویری و صوتی.
    \item \textbf{سرور:} استفاده از Tkinter برای رابط کاربری و OpenCV برای ضبط تصاویر از وب‌کم، همراه با PyAudio برای ضبط صوت.
    \item \textbf{پیش‌نیازها:} Python OpenCV, PyAudio, Pillow,
    \item \textbf{استفاده:} اجرای اسکریپت \texttt{server.py}، کلیک بر روی "شروع سرور" برای راه‌اندازی و ضبط تصاویر و صدا. فایل‌های دریافتی در دایرکتوری‌های \texttt{captured\_images\_server} و \texttt{captured\_audio\_server} ذخیره می‌شوند.
    \item \textbf{کلاینت:} مسئول ارسال داده‌های تصویری و صوتی به سرور. به طور جداگانه بر اساس آدرس IP و شماره پورت سرور پیاده‌سازی می‌شود.
    \item \textbf{نکات:} اطمینان از پیکربندی‌های صحیح شبکه برای ارتباط؛ تنظیم آدرس IP و شماره پورت‌ها در کد بر اساس نیاز.
\end{itemize}

% Skills Section
\section{مهارت‌ها}

\begin{onecolentry}
    \textbf{زبان‌های برنامه‌نویسی:} Python، C++، Java، C
\end{onecolentry}

\vspace{0.1 cm}

\begin{onecolentry}
    \textbf{توسعه وب:} HTML، CSS، JavaScript، React، Django
\end{onecolentry}

\vspace{0.1 cm}

\begin{onecolentry}
    \textbf{یادگیری ماشین:} کتابخانه‌های پایتون مانند NumPy، pandas، scikit-learn، TensorFlow، Keras,
\end{onecolentry}

\vspace{0.1 cm}

\begin{onecolentry}
    \textbf{پردازش تصویر:} کتابخانه‌های پایتون مانند OpenCV، Pillow، scikit-image,
\end{onecolentry}

\vspace{0.1 cm}

\begin{onecolentry}
    \textbf{داده‌کاوی:} آشنایی با تکنیک‌های تحلیل و داده‌کاوی (مانند \texttt{pandas}، \texttt{NumPy}، \texttt{BeautifulSoup})
\end{onecolentry}

\vspace{0.1 cm}

\begin{onecolentry}
    \textbf{شبکه:} درک پروتکل‌های شبکه، TCP/IP و تجربه با برنامه‌نویسی سوکت
\end{onecolentry}

\vspace{0.1 cm}

\begin{onecolentry}
    \textbf{لینوکس:} آشنایی ابتدایی با مفاهیم و دستورات لینوکس
\end{onecolentry}

\vspace{0.1 cm}

\begin{onecolentry}
    \textbf{سایر:} Git , الگوریتم‌ها، ساختارهای داده، حل مسئله
\end{onecolentry}

\end{document}
