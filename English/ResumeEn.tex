\documentclass[10pt, a4paper]{article}

% Packages:
\usepackage[
    ignoreheadfoot, 
    top=2 cm,
    bottom=2 cm,
    left=2 cm,
    right=2 cm,
    footskip=1.0 cm
]{geometry} % for adjusting page geometry
\usepackage{titlesec} % for customizing section titles
\usepackage{tabularx} % for making tables with fixed width columns
\usepackage{array} % tabularx requires this
\usepackage[dvipsnames]{xcolor} % for coloring text
\usepackage{enumitem} % for customizing lists
\usepackage{fontawesome5} % for using icons
\usepackage{amsmath} % for math
\usepackage[
    pdftitle={Erfan Habibi Ehsani's CV},
    pdfauthor={Erfan Habibi Ehsani},
    colorlinks=true,
    urlcolor=black
]{hyperref} % for links, metadata and bookmarks
\usepackage{lastpage} % for getting the total number of pages
\usepackage{changepage} % for one column entries (adjustwidth environment)
\usepackage{paracol} % for two and three column entries
\usepackage{needspace} % for avoiding page break right after the section title
\usepackage{iftex} % check if engine is pdflatex, xetex or luatex

% Font setup - New Times Roman
\usepackage{newtxtext,newtxmath} % Times New Roman font

% Some settings:
\raggedright
\AtBeginEnvironment{adjustwidth}{\partopsep0pt} % remove space before adjustwidth environment
\pagestyle{empty} % no header or footer
\setcounter{secnumdepth}{0} % no section numbering
\setlength{\parindent}{0pt} % no indentation
\setlength{\topskip}{0pt} % no top skip
\setlength{\columnsep}{0.15cm} % set column separation
\pagenumbering{gobble} % no page numbering

\titleformat{\section}{\needspace{4\baselineskip}\bfseries\large}{}{0pt}{}[\vspace{1pt}\titlerule]

\titlespacing{\section}{
    -1pt
}{
    0.3 cm
}{
    0.2 cm
}

\renewcommand\labelitemi{$\vcenter{\hbox{\small$\bullet$}}$} % custom bullet points

% Define missing environments
\newenvironment{header}{
    \centering
}{\par}

\newenvironment{twocolentry}[1]{
    \begin{tabularx}{\textwidth}{Xr}
}{\end{tabularx}}

\newenvironment{onecolentry}{
    \begin{adjustwidth}{0pt}{0pt}
}{\end{adjustwidth}}

\newenvironment{highlights}{
    \begin{itemize}[leftmargin=*]
}{\end{itemize}}

% Begin Document
\begin{document}

% Header
\begin{header}
    \fontsize{25 pt}{25 pt}\selectfont Erfan Habibi Ehsani

    \vspace{5 pt}

    \normalsize
    % Tehran, Iran%
    % \kern 5.0 pt%
    % \textbar%
    \kern 5.0 pt%
    \href{mailto:erfan.habibi.ehsani@gmail.com}{erfan.habibi.ehsani@gmail.com}% 
    \kern 5.0 pt%
    \textbar%
    \kern 5.0 pt%
    \href{https://www.linkedin.com/in/erfanhabibi/}{linkedin.com/in/erfanhabibi}% 
    \kern 5.0 pt%
    \textbar%
    \kern 5.0 pt%
    \href{https://github.com/Erfanhabibi}{github.com/Erfanhabibi}%
    \textbar%
    \kern 5.0 pt%
    \href{tel:+989144975069}{+98 914 497 5069}
\end{header}

\vspace{10 pt}

% Education Section
\section{Education}

\begin{tabularx}{\textwidth}{Xr}
    \textbf{B.Sc. in Mathematics and Computer Science}, Amirkabir University of Technology, Tehran, Iran & 2020 -- Present \\
\end{tabularx}

\vspace{0.1 cm}

\textbf{Expected Graduation:} 2025 \\
\textbf{Relevant Coursework:} Data Structures, Algorithms, Machine Learning, Advanced Programming

% % Experience Section
% \section{Experience}

% \begin{tabularx}{\textwidth}{Xr}
%     \textbf{Personal and Academic Coding Projects} & 2021 -- Present \\
% \end{tabularx}

% \vspace{0.1 cm}

% \textbf{Highlights:}
% \begin{itemize}
%     \item Developed a Python application utilizing OpenCV for real-time image processing,
%           enabling features such as object detection and image filtering.
%     \item Implemented a data mining project using Python, extracting valuable insights
%           from large datasets with libraries like pandas and NumPy to inform
%           decision-making.
%     \item Created interactive web applications using Java and JavaScript, enhancing user
%           experience through dynamic content and responsive design.
% \end{itemize}

% Projects Section
\section{Projects}

\begin{tabularx}{\textwidth}{Xr}
    \textbf{Server-Client Communication System} & \href{https://github.com/Erfanhabibi/python-socket.git}{github.com/Erfanhabibi/python-socket} \\
\end{tabularx}

\vspace{0.1 cm}

\textbf{Highlights:}
\begin{itemize}
    \item Implemented a server-client communication system using Python's sockets for
          transferring images and audio data.
    \item \textbf{Server:} Utilizes Tkinter for the GUI and OpenCV for capturing images from the webcam, along with PyAudio for recording audio.
    \item \textbf{Prerequisites:} Python 3.x, OpenCV , PyAudio, Pillow .
    \item \textbf{Usage:} Run the \texttt{server.py} script, click "Start Server" to initialize, and capture images and record audio. Received files are saved in the \texttt{captured\_images\_server} and \texttt{captured\_audio\_server} folders.
    \item \textbf{Client:} Responsible for sending images and audio data to the server. Implemented separately based on the server's IP address and port numbers.
    \item \textbf{Notes:} Ensure proper network configurations for communication; adjust the IP address and port numbers in the code as needed.
\end{itemize}

\begin{tabularx}{\textwidth}{Xr}
    \textbf{Message Broker System} & \href{https://github.com/Erfanhabibi/MessageBroker}{github.com/Erfanhabibi/MessageBroker} \\
\end{tabularx}

\vspace{0.1 cm}

\textbf{Highlights:}
\begin{itemize}
    \item Developed a Message Broker system in C\# to manage data transfer between producers and consumers while ensuring message order and persistence.
    \item \textbf{Endpoints:} Designed endpoints for producers to send messages and consumers to receive them.
    \item \textbf{Multithreading:} Implemented thread management to allow producers and consumers to define their required number of threads dynamically.
    \item \textbf{Message Persistence:} Ensured data durability by storing messages in files to prevent data loss in case of server failures or restarts.
    \item \textbf{Plugin System:} Designed a decoupled architecture using interfaces and attributes, enabling third-party developers to extend producer and consumer functionalities dynamically.
    \item \textbf{Retry Mechanism:} Implemented an automatic retry mechanism for failed message deliveries, ensuring messages are resent when the server is down and retried at configured intervals.
    \item \textbf{Logging System:} Developed a logging mechanism to track key operations such as message sending, receiving, storing, and recovery, with configurable log levels (Info, Warning, Error).

\end{itemize}

% Skills Section
\section{Skills}

\begin{onecolentry}
    \textbf{Programming Languages:} Python, C++, Java, C, C\#
\end{onecolentry}


\vspace{0.1 cm}

\begin{onecolentry}
    \textbf{Machine Learning:} Python libraries such as NumPy, pandas, scikit-learn, TensorFlow, Keras
\end{onecolentry}

\vspace{0.1 cm}

\begin{onecolentry}
    \textbf{Image Processing:} Python libraries such as OpenCV, Pillow, scikit-image
\end{onecolentry}

\vspace{0.1 cm}

\begin{onecolentry}
    \textbf{Data Mining:} Knowledge of data analysis and mining techniques (e.g., \texttt{pandas}, \texttt{NumPy}, \texttt{BeautifulSoup})
\end{onecolentry}

\vspace{0.1 cm}

\begin{onecolentry}
    \textbf{Networking:} Understanding of networking protocols, TCP/IP, and experience with socket programming
\end{onecolentry}

\vspace{0.1 cm}

\begin{onecolentry}
    \textbf{Linux:} Basic knowledge of Linux concepts and commands
\end{onecolentry}

\vspace{0.1 cm}

\begin{onecolentry}
    \textbf{Other:} Git, Algorithms, Data Structures, Problem Solving
\end{onecolentry}

\end{document}
